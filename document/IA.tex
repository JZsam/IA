\documentclass{article}
\usepackage{graphicx}
\usepackage{makecell}
\usepackage{listings}
\usepackage{color}
\usepackage{apacite}%Also use in Capstone
\usepackage[utf8]{inputenc}
\usepackage[letterpaper, margin=1.25in]{geometry}
\pagenumbering{arabic}
\title{IB Computer Science IA | Attendance System for the Marching Band}
\author{Jacob Samurin}

\definecolor{dkgreen}{rgb}{0,0.6,0}
\definecolor{gray}{rgb}{0.5,0.5,0.5}
\definecolor{mauve}{rgb}{0.58,0,0.82}

\lstset{frame=tb,
	language=Java,
	aboveskip=3mm,
	belowskip=3mm,
	showstringspaces=false,
	columns=flexible,
	basicstyle={\small\ttfamily},
	numbers=left,
	numberstyle=\tiny\color{gray},
	keywordstyle=\color{blue},
	commentstyle=\color{dkgreen},
	stringstyle=\color{mauve},
	breaklines=true,
	breakatwhitespace=true,
	tabsize=3
}

\begin{document}
%TODO Comment this out when turn in
\maketitle
\newpage
\tableofcontents
\newpage
%To here
\section{Criterion A: Planning}
\subsection{Defining the Problem}
The problem of Mr. Todd Fessler (my client) is that for our marching band class there is no good and efficient way of taking attendance. The way of taking attendance right now is that the “Drum Majors” who are the overall leaders in the marching band, go around and ask each row for their attendance, and it took a very long time to take attendance.
\subsection{Rational for Proposed Solution}
My solution will make it possible for the leaders of each row to take attendance then the Drum Majors won’t have to go row by row. This will also skip the Drum Major step completely since my client will have direct accesses to the app. This will make it easier for everyone and faster, so we can have a longer rehearsal times.
\subsection{Success Criterion}
\begin{center}
	\begin{tabular}{|l|}
		\hline
		Different classes each level of leaders\\
		\hline
		A web app will be created add the different levels can edit different kinds of pages\\
		\hline
		Make it accessible from a phone\\
		\hline
		Make it accessible from an iPad\\
		\hline
		Make it accessible from a laptop/computer\\
		\hline
		Have the sever running the web app set up\\
		\hline
	\end{tabular}
\end{center}
\newpage
\section{Criterion B: Solution Overview}
\subsection{Sketch}
\subsection{UML Diagram}
This is the basic outline of the different classes and data types \\ \\
%TODO add toString() to Bandie
%TODO move update band to Band class
\includegraphics[width=6in]{IA UML and Flowchart-UML.jpg}
\subsection{Flowchart}
This is the basic flow of the \verb|updateBand| method \\ \\
\includegraphics[width=6in]{IA UML and Flowchart-Flowchart.jpg}
\subsection{Pseudocode}
For the pseudocode I will be writing out the \verb|updateBand| method
\begin{verbatim}
				item = read in a file for every section that stops in a comma
				while there is a next item
					if tem%6 equals 0 
						spot = item
						if its k-row
							row = first 2 letter
						else
							row = first letter
					if temp%6 equals 1
						name = item + " "
					if temp%6 equals 2
						name += item
					if temp%6 equals 3
						section = item
					if temp%6 equals 4
						grade = item
					if temp%6 equals 5
						new object t of type Top(spot, row, name, section, grade, item)
					temp++
\end{verbatim}
\subsection{Development Plan}
This is a plan for how to create the final product
\begin{itemize}
	\item Create the Bandie class and the other levels of authorization all the way to Director/Admin
	\item Write the Row class with the hash table for the whole row’s attendance and a linked list for the Bandies in the Row
	\item Create the Band class with the linked list for the directors and another linked list for the rows
	\item Create the Attendance class to store the past and current attendance for each Bandie
	\item Be able to upload a new CSV file to upload the structure of the band
	\item Store past attendance for each Bandie in a CSV file
	\item Make a web app that will ...
		\begin{enumerate}
			\item Have a login screen where any level above Bandie has a password
			\item Have the different screens for Bandies, Squad Leaders, and Drum Majors.
			\item For Directors, they will also have a Statistics page with a tagline at the top(refer to the Sketch above)
		\end{enumerate}
\end{itemize}
\subsection{Test Cases}
\resizebox{6in}{!}{
	\begin{tabular}{|p{3in}|p{3in}|}
		\hline
		\LARGE Case & \LARGE Outcome\\
		\hline \hline \\
		\Large Logging into a higher level account &
		\begin{itemize}
			\item An incorrect password will send errors to the user.
			\item A correct password will let the user through.
			\item An invalid password(i.e. sending in the wrong data type) will send an error to the user 
		\end{itemize}
		\\
		\hline \\
		\Large Uploading a spreed sheet file &
		\begin{itemize}
			\item If the file is not a CSV file then will pass an error to the user "Improper File type"
			\item If the file is a CSV but is not properly formatted it will pass an error to the user "Improper formatted"
			\item If the file is a CSV and the first line is properly formatted it will upload the file
		\end{itemize}
		\\
		\hline %\\
		% \Large Input for the attendace &
		% \begin{itemize}
		% 	\item
		% \end{itemize}
		% \\
		% \hline
	\end{tabular}
}
\subsection{Record of Tasks}
\resizebox{6in}{!}{
	\begin{tabular}{| c | p{1.5in} | p{1.5in} | p{1in} | p{1in} | c |}
		\hline
		Task Number & Planned Action & Planned Outcome & Time Estimated (Minutes) & Target Completion Date & Criterion\\
		\hline
		1 & Brainstorming with my client & An idea for the project & 30 & May 5, 2022 & A\\
		\hline
		2 & Interview & Get to know what my client wants & 8.5 & May 24, 2022 & A\\
		\hline
		3 & Write a draft proposal & proposed to teacher & 60 & May 26, 2022 & A\\
		\hline
		4 & Write a proposal & Re-proposed  to teacher & 75 & Aug 22, 2022 & A\\
		\hline
		5 & make a Record of Tasks & Organize a timeline & 15 & Aug 23, 2022 & B\\
		\hline
		6 & Second Interview & Get a better Idea from my client & 7 & Sep 1, 2022 & A\\
		\hline
		7 & Create UML diagram & Get a UML diagram & 90 & Oct 20, 2022 & B\\
		\hline
		8 & Working on criterion B & Complete the outline of the criterion B & 20 & Oct 21, 2022 & B\\
		\hline
		9 & Third Interview  & Get a better idea of the statistics section & 6 & Oct 26, 2022 & A\\
		\hline
		10 & Create drawing UI & To put my idea on to paper & 45 & Nov 5, 2022 & B\\
		\hline
		11 & Re-setup the document & Make it easier to work on the document & 180 & Nov 8, 2022 & A/B\\
		\hline
		12 & added a case to the Test Cases section & Work on a part needed for the finishing Criterion B& 45 & Nov 9, 2022 & B\\
		\hline
		13 & added a case to the test Cases section & work on a part needed for the finishing criterion b& 15 & Nov 9, 2022 & B\\
		\hline
		14 & fix the UML and finish Flowchart and Pseudocode & fixed UML and finished Flowchart and Pseudocode & 90 & Dec 19, 2022 & B\\
		\hline
		15 & writing the hash table section & started work on the writing part of criterion C & 30 & Jan 8, 2023 & C\\
		\hline
		16 & Codeing to complete criterion c & complete the coding for criterion c & 240 & Feb 4, 2023 & C\\
		\hline
	\end{tabular}
}
\newpage
\section{Criterion C: Development}
\subsection{Program Structure}
The main parts of my program are the \verb|App.java|, \verb|Bandie.class|, \verb|Row.java|, and \verb|Band.java|. The \verb|App| class, this has my main method and the details for my basic Terminal User Interface(TUI). The \verb|Bandie| class is used as a base for my inheritance for every member of the band. \verb|Row.java| is using a LinkedList\cite{linkedList} of \verb|Bandie| to simulate what row would look link in real life. The final class \verb|Band| is working like a HashTable\cite{hashTable} because it contains a LinkedList\cite{linkedList} of \verb|Row|. This make \verb|Band.java| act like a band block would even have a dynamic amount of people on each row by using the \verb|Row| object.\\
\includegraphics[width=3in]{fileStructure.png}
\subsection{Techniques Used}
\begin{itemize}
	\item Inheritance
	\item Polymorphism
	\item Encapsulation
	\item File Reading and Writing
	\item HashTable\cite{hashTable}
	\item LinkedList\cite{linkedList}
	\item HashMaps\cite{hashMap}?(Ask DK)%TODO ask DK
	\item Singleton Class?(Ask DK)%TODO Ask DK
\end{itemize}
\subsubsection{Inheritance}
I'm using inheritance as a way to easily have the same base for all my users. Each user (except Directors) are inheriting aspects from \verb|Bandie| and other objects. The major point of using this was to make is easy to instantiate any of the users in my program. Here is an example from my \verb|Director| class.
\newpage
\begin{lstlisting}
// Director.java
		Bandie current = new Bandie(); // This is setting up a default
		Director direct  = new Director();
		LinkedList<Row> newBlc = new LinkedList<Row>();
		boolean dir = false;
		Scanner s = new Scanner(f);
		String name = "";
		s.useDelimiter(",");
		int count=0,realCount=0;
		if(topLine)
			s.nextLine();
		while(s.hasNext()){
			String item = s.next();
			switch(count%6){
				case 0:
					switch(item){ //Here is were we actualy where we can make this a any other type of user
						case "director":
							dir=true;
							break;
						case "officer":
							current = new Officer(); //This will change current into an Officer
							break;
						case "leader":
							current = new SquadLeader();//This will change current into an Squad Leader
							break;
						case "bandie":
							current= new Bandie();//This will keep current the same as the default of Bandie
							break;
					}
					break;
\end{lstlisting}
On line 2 we set a default and the left hand side of the instantiate, down at line 16 we have a switch case which well make the \verb|current| on line 2 into any other type of user from lines 21 to 27. 
\subsubsection{Polymorphism}
Polymorphism is when you have different classes that know what they are and what they can do. I'm using this when I'm calling the \verb|takeAttendance| and the \verb|takeRowAttendance| between when \verb|Director|, \verb|Officer|, or \verb|SquadLeader| call it.
\begin{lstlisting}
//SquadLeader.java
public void takeRowAttendance(char[] a){
	Band.getRow(super.getRow()).updateRecord(this);
	for (int i = 0; i < Band.getRow(super.getRow()).size(); i++) {
		takeAttendance((""+super.getRow())+(i+1), a[i]);
	}
}
public void takeAttendance(String spott, char a
	Band.getBandie(spott).setAttendance(a);
}

//Office.java
public void takeRowAttendance(char row, char[] a){
	Band.getRow(row).updateRecord(this);
	for (int i = 0; i < Band.getRow(row).size(); i++) {
		super.takeAttendance((""+row)+(i+1), a[i]);
	}
}

//Director.java
public void takeRowAttendance(char row, char[] a){
	Band.getRow(row).updateRecord(this);
	for (int i = 0; i < Band.getRow(row).size(); i++) {
		takeAttendance((""+row)+(i+1), a[i]);
	}
}
public void takeAttendance(String spott, char a){
	Band.getBandie(spott).setAttendance(a);
}
\end{lstlisting}
This example is showing the difrance between the \verb|takeRowAttendance| method in \verb|SquadLeader| on line 1 and \verb|Offcer|/\verb|Director| on line 12 and 20. The \verb|takeRowAttendance| on line 2 us more so used when the row is predefind like in \verb|SquadLeader.java| where they can only change their row. But the \verb|takeRowAttendance| on line 13 and 21 can change any row that is asked.
\subsubsection{Encapsulation}
Encapsulation is the practice of keeping everything contained in one object and using accessor methods to get values. I use this in the \verb|Row| class because I don't have access to the LinkedList\cite{linkedList} it's self, but I can get the size trough the \verb|size()| method. Here is that example
\begin{lstlisting}
//Attendance.java
package com.jzdoot.IA;
import java.util.Date;
import java.util.HashMap;

public class Attendance{
	private HashMap<Date,Character> record;
	//NOTE the currentAttendace is now in the bandie class

	public void updateAttendance(char attendance){
		record.put(new Date(), attendance);
	}
}
\end{lstlisting}
In this example the HashMap\cite{hashMap} on line 7 is private and the method \verb|updateAttendance| on line 10 is used to set the HashMap\cite{hashMap} instead of changing the HashMap\cite{hashMap} directly.
\subsubsection{File Reading and Writing}
File reading and writing is a way for a program to read in a file and output out to a file. I'm using it in one main way, in the \verb|Director| class there is a method called \verb|updateBand| it is used to update the block structure easily by using a \verb|.csv| file because it's stored in plain text but can be edited by any spreadsheet software. I use this to store the band and to reset it when the program gets closed. To store this copy this file to a new file in a saved directory\cite{moveFileStackoverflow}.
\subsubsection{HashTable}
HashTables\cite{hashTable} are arrays of LinkedLists\cite{linkedList}. I'm using this in a different way I made a LinkedList\cite{linkedList} of \verb|Row|, but \verb|Row| is a LinkedList\cite{linkedList} of \verb|Bandie| this works in the same general way as a HashTable\cite{hashTable}. The way I'm using this "HashTable" is to store the structure of the band and each row will be able to be dynamic and so will the number of rows which is why I didn't use just a normal HashTable\cite{hashTable}.
\newpage
\section{Criterion E: Evaluation}
\newpage
\section{Appendix A: Interview}
Interview conducted in person
Interview 1 initial information gathering

Student - Basically my plan right now for the for the program is there's going to be different tiers. So, there’s going to be a squad leader tier, a drum major tier, and then a director/admin tier. Squad leaders will be able to change attendance within the row, for Drum Major will be able to change attendance for the entire band and then for directors they can change the attendance for the entire band and on top of that there's some statistics like what rows haven't taken attendace yet, who turned into attendance for which row, and then maybe some row of the week stuff like most improved week to week, move most improved within the week, or best overall within the week or for the entire year. I’ll be able to collect all that kind of data just from attendance.
Client – That’s awesome. So would it be like you're thinking, like you said squad leaders have control over their rows. 
Student - and then drum majors the entire band and then you guys can take this for the entire band and then you guys can do attendance for the entire band and then on top of that you can look at statistics and stuff
Client - Is there any way once like for example of this squad leaders have submitted their attendance for the day is there a way for them to update it?
Student - I mean I could definitely implement that.
Client - or like a way for them to update it until the end of class. Somebody shows up that's the issue I think yeah if somebody shows up late on excused then they're marked absent for the whole day but we need to know if they were late that day or if they were asking because that changes 
Student –I can probably set up a window of time like every morning from like 7:30 to 8:30 maybe even earlier I can set it to like 7 like to whenever time the period ends nowadays. Talking about some basic structure %2:04
\newpage
\section{Appendix B: References}
% TODO Write in the sources for the java docs
% test\cite{java17}
\bibliography{IA}{}
\bibliographystyle{apacite}
\end{document}
